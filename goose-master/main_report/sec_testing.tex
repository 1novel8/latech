\section{ПРОГРАММА И МЕТОДИКА ИСПЫТАНИЙ}
\label{sec:testing}

В данном разделе рассмотрено тестирование разработанной системы.
Его необходимо проводить перед выпуском продукта или системы
с целью выявления ошибок и их устранения, для проверки соответствия продукта
заявленным требованиям, оценки качества работы разработчиков,
получения информации о сложности системы и ее характеристиках. Тестирование
обеспечивает безопасность кода при командной работе, помогает в создании наилучшей
архитектуры, улучшает качество кода, упрощает исправление ошибок и экономит денежные
средства организации.

Модуль приема и обработки GOOSE-пакетов тщательно тестировался в процессе разработки.
Как было описано в разделе~\ref{sec:func}, в системе имеются как функции настроек
и взаимодействия с другими модулями, так и функции, занимающиеся логикой принятия и
обработки пакетов. Важная часть тестирования отводилась и модулю конфигурации, а именно
работа с \cid-файлами и корректное разложение в память полученных данных.

Для проведения модульного тестирования используется библиотека Unity. Она предоставляет
большое количество способов проверки корректности значений и написана с использованием
стандарта C99.

Предоставление возможности использования функций других модулей системы без реализации
предоставляет библиотека CMock от разработчиков Unity. Она написана на языках программирования
C и Ruby и способна генерировать реализации функций по их прототипам. Библиотека
производит создание необходимых реализаций в соответствии со
стандартом C99, а также имеет гибкие
возможности по установлению соответствия пользовательских типов с типами стандартной библиотеки
языка C.

Проверка работоспособности системы была проведена с помощью большого количества тестов, поэтому на рассмотрение вынесено тестирование самых важных алгоритмов и функций.

\subsection{Тесты модуля \moduleCfg}

Первая группа тестов посвящена модулю \moduleCfg. Важным алгоритмом системы является
функция разбора \cid-файла \lstinline{conf_parseGooseIntAddr_private}.
Ее тестирование производилось с помощью описанных далее функций.

Тест \lstinline{test_parseGooseIntAddr_confusedParam} принимает некорректные строки \cid-файла с перепутанной последовательностью правильно заполненных элементов \lstinline{source}, \lstinline{mac}, \lstinline{goId}, \lstinline{appId}, \lstinline{confRev}, \lstinline{vars}. В тесте создается корректный вариант строки и вызывается проверяемая функция разбора строки \cid-файла. Все входные неправильные строки теста сравниваются с ней, в случае совпадения тест проходит, при несоответствии~-- возвращает номер строки и вспомогательное сообщение  об ошибке.

Тест \lstinline{test_parseGooseIntAddr_errorBrackets} принимает некорректные строки \cid-файла с ошибочным количеством символов <<\}>> или <<\{>>. Также скобки могут быть расставлены в неподходящем порядке, что тоже является ошибкой. В тесте происходит вызов проверяемой функции разбора строки \cid-файла. Проверяется возвращаемое значение на \lstinline{NULL}: если совпадает, то тест возвращает номер строки и вспомогательное сообщение об ошибке, в обратном случае он проходит.

Тест \lstinline{test_parseGooseIntAddr_good} принимает правильную строку \cid-файла. Создается корректный вариант строки и вызывается проверяемая функция разбора строки \cid-файла в тесте. Входная правильная строка сравнивается с созданной, в случае совпадения тест проходит, при несоответствии~-- возвращает номер строки и вспомогательное сообщение  об ошибке.

Тест \lstinline{test_parseGooseIntAddr_errorSource} принимает некорректные строки \cid-файла с ошибочным параметром \lstinline{source}, все остальные значения верны. В тесте происходит вызов функции разбора строки \cid-файла. Проверяется возвращаемое значение на \lstinline{NULL}: если совпадает, то тест возвращает номер строки и вспомогательное сообщение об ошибке, в обратном случае он проходит.

Тест \lstinline{test_parseGooseIntAddr_correctSource} принимает правильные строки \cid-файла с всевозможным значением параметра \lstinline{source}, все остальные значения верны тоже. В тесте создается корректный вариант строки для каждого варианта принимаемой строки и вызывается функция разбора строки \cid-файла. Все входные строки сравниваются с созданными строками в тесте. В случае совпадения он проходит, при несоответствии~-- возвращает номер строки и вспомогательное сообщение  об ошибке.

Тест \lstinline{test_parseGooseIntAddr_errorMacAddress} принимает некорректные строки \cid-файла с ошибочным параметром \lstinline{mac}, все остальные значения верны. В тесте происходит вызов функции разбора строки \cid-файла. Проверяется возвращаемое значение на \lstinline{NULL}: если совпадает, то тест возвращает номер строки и вспомогательное сообщение об ошибке, в обратном случае он проходит.

Тест \lstinline{test_parseGooseIntAddr_errorGoId} принимает некорректные строки \cid-файла с ошибочным параметром \lstinline{goId}, все остальные значения верны. В тесте происходит вызов функции разбора строки \cid-файла. Проверяется возвращаемое значение на \lstinline{NULL}: если совпадает, то тест возвращает номер строки и вспомогательное сообщение об ошибке, в обратном случае он проходит.

Тест \lstinline{test_parseGooseIntAddr_errorConfRev} принимает некорректные строки \cid-файла с ошибочным параметром \lstinline{confRev}, все остальные значения верны. В тесте происходит вызов функции разбора строки \cid-файла. Проверяется возвращаемое значение на \lstinline{NULL}: если совпадает, то тест возвращает номер строки и вспомогательное сообщение об ошибке, в обратном случае он проходит.

Тест \lstinline{test_parseGooseIntAddr_correctConfRev} принимает правильную строку \cid-файла со всевозможным значением параметра \lstinline{confRev}. Создается корректный вариант строки для каждой из входных строк и вызывается функция разбора строки \cid-файла в тесте. Входная строка сравнивается с созданной, в случае совпадения тест проходит, при несоответствии~-- возвращает номер строки и вспомогательное сообщение  об ошибке.

Тест \lstinline{test_parseGooseIntAddr_confRevNull} принимает правильную строку \cid-файла c нулевым значением параметра \lstinline{confRev}. Создается корректный вариант строки и вызывается функция разбора строки \cid-файла в тесте. Входная строка сравнивается с созданной, в случае совпадения тест проходит, при несоответствии~-- возвращает номер строки и вспомогательное сообщение об ошибке.

Тест \lstinline{test_parseGooseIntAddr_errorFormating} принимает некорректные строки \cid-файла с ошибочным количеством символов <<,>>, <<.>> или <<;>>. В тесте происходит вызов функции разбора строки \cid-файла. Проверяется возвращаемое значение на \lstinline{NULL}: если совпадает, то тест возвращает номер строки и вспомогательное сообщение об ошибке, в обратном случае он проходит.

Тест \lstinline{test_parseGooseIntAddr_errorAppId} принимает некорректные строки \cid-файла с ошибочным параметром \lstinline{appId}, все остальные значения верны. В тесте происходит вызов функции разбора строки \cid-файла. Проверяется возвращаемое значение на \lstinline{NULL}: если совпадает, то тест возвращает номер строки и вспомогательное сообщение об ошибке, в обратном случае он проходит.

Тест \lstinline{test_parseGooseIntAddr_correctAppId} принимает правильную строку \cid-файла со всевозможным значением параметра \lstinline{appId}. Создается корректный вариант строки для каждой из входных строк и вызывается функция разбора строки \cid-файла в тесте. Входная строка сравнивается с созданной, в случае совпадения тест проходит, при несоответствии~-- возвращает номер строки и вспомогательное сообщение об ошибке.

Тест \lstinline{test_parseGooseIntAddr_appIdNull} принимает правильную строку \cid-файла c нулевым значением параметра \lstinline{appId}. Создается корректный вариант строки и вызывается функция разбора строки \cid-файла в тесте. Входная строка сравнивается с созданной, в случае совпадения тест проходит, при несоответствии~-- возвращает номер строки и вспомогательное сообщение об ошибке.

Тест \lstinline{test_parseGooseIntAddr_errorVars} принимает некорректные строки \cid-файла с ошибочным количеством параметров \lstinline{vars} или неправильными значениями этих параметров. В тесте происходит вызов функции разбора строки \cid-файла. Проверяется возвращаемое значение на \lstinline{NULL}: если совпадает, то тест возвращает номер строки и вспомогательное сообщение об ошибке, в обратном случае он проходит.

Тест \lstinline{test_parseGooseIntAddr_vars1} принимает правильную строку \cid-файла с одним параметром \lstinline{vars} с корректным значением. В тесте происходит вызов функции разбора строки \cid-файла и создание строки для сравнения. Входная строка сравнивается с созданной, в случае совпадения тест проходит, при несоответствии~-- возвращает номер строки и вспомогательное сообщение об ошибке.

Тест \lstinline{test_parseGooseIntAddr_vars2} является аналогом предыдущего
теста, как и \lstinline{test_parseGooseIntAddr_vars3}. Они различаются только
количеством принимаемых параметров \lstinline{vars} и проверяют корректность
обработки нескольких похожих блоков.

Вспомогательной функцией модуля конфигурации является функция преобразования MAC-адреса в правильный формат~-- \lstinline{test_convertMacToCorrectFormat}. Ниже описаны тесты для ее тестирования.

Тест \lstinline{test_convertMacToCorrectFormat_correct} принимает правильные MAC-адреса для преобразования в нужный формат. Они содержат корректные символы, строки полные, без лишних элементов. Тест выделяет буфер под новый преобразованный адрес и вызывает функцию конвертации. Проверяется возвращаемое значение на \lstinline{true}: если произошла ошибка преобразования, то тест возвращает номер строки и вспомогательное сообщение об ошибке, в обратном случае он проходит.

Тест \lstinline{test_convertMacToCorrectFormat_error} принимает некорректные MAC-адреса для преобразования в нужный формат. Они содержат ошибочные символы, неполные строки, лишние элементы. Тест выделяет буфер под новый преобразованный адрес и вызывает функцию конвертации. Проверяется возвращаемое значение на \lstinline{true}: если произошла ошибка преобразования, то тест возвращает номер строки и вспомогательное сообщение об ошибке, в противном случае он проходит.

Тест \lstinline{test_convertMacToCorrectFormat_upper} принимает корректные MAC-адреса для преобразования в нужный формат, состоящие только из прописных букв. Он выделяет буфер под новый преобразованный адрес и вызывает функцию конвертации. Проверяется возвращаемое значение на \lstinline{true}: если произошла ошибка преобразования, то тест возвращает номер строки и вспомогательное сообщение об ошибке, в противном случае он проходит. Также проверяется преобразованный адрес: в тесте создаются ожидаемые результаты преобразованных MAC-адресов и производится сравнение с результатом работы функции.

Тесты \lstinline{test_convertMacToCorrectFormat_goodNumber} и \lstinline{test_convertMacToCorrectFormat_lower} работают аналогично предыдущему: принимают корректные MAC-адреса для преобразования в нужный формат, состоящие только из  цифровых значений или  строчных букв соответственно.

\subsection{Тесты модуля \moduleRecvPackets}

Модуль \moduleRecvPackets\ протестирован особенно тщательно в связи с возлагаемой на него логической нагрузкой и важностью занимаемого места в системе. Проверена как внутренняя работа, так и внешнее взаимодействие с другими модулями системы. Некоторые тесты невозможно было написать только для этого модуля, поэтому они охватывают сразу несколько модулей.

\subsubsection{Основная бизнес-логика модуля}

Тест \lstinline{test_invalidDstMacAddr} открывает вспомогательную тестовую очередь, создает буфер с отправляемым модулю пакетом для дальнейшего сравнения.
Созданный GOOSE-пакет с некорректным  MAC-адресом назначения (такой адрес, который не совпадает с заявленной конфигурацией ожидаемого пакета) посылается в модуль. По завершении теста ожидается, что память, в которую модуль должен был разложить данные, не изменилась. Если данное условие выполняется, то тест завершается успехом, в обратном случае ожидается ошибка.

Тест \lstinline{test_correctDstMacAddr} открывает вспомогательную тестовую очередь, создает буфер с отправляемым модулю пакетом для дальнейшего сравнения.
Созданный GOOSE-пакет с корректным  MAC-адресом назначения посылается в модуль. По завершении теста ожидается, что память, в которую модуль должен был разложить данные, изменилась и соответствует заявленному пакету. Если данное условие выполняется, то тест завершается успехом, в обратном случае ожидается ошибка.

Выше описаны два теста для MAC-адреса назначения. Аналогичные тесты были проведены для следующих параметров как \lstinline{goCbRef} и \lstinline{appId}.
Тесты \lstinline{test_invalidGoCbRef}, \lstinline{test_correctGoCbRef}, \lstinline{test_invalidAppId} и \lstinline{test_correctAppId} по завершении проверяют наличие или отсутствие изменений в памяти и возвращают соответствующий результат.

Тест \lstinline{test_correctPacket} принимает пакет со всеми правильно заполненными полями и с максимальной заполненностью. Основная задача данного теста~-- выяснить, справляется ли модуль с пакетами максимального размера. Проверка происходит по количеству измененных байт в памяти, а также по наличии новых данных, которые соответствуют заявленным.

Тест \lstinline{test_confuseNestings} работает с корректным пакетом, но с перепутанными между собой уровнями вложенности. Проверяется, способен ли модуль правильно обрабатывать такие пакеты. Это происходит по количеству измененных байт в памяти, а также по наличии новых правильно разложенных данных, которые соответствуют заявленным.

\subsubsection{Входные параметры модуля}

У каждого модуля системы есть входные параметры, такие как приоритет потоков, глубина очередей, максимальный размер передаваемых и принимаемых через очередь сообщений. В следующей подгруппе тестов происходит проверка этих параметров.

Тест \lstinline{test_gooseInput_allParamNULL} принимает все вышеописанные параметры нулевыми. Вызывается функция инициализации модуля, проверяется, что система не сломалась и функция вернула \lstinline{GOOSE_INPUT_RET_CODE_INIT_ERROR}.

Тест \lstinline{test_gooseInput_messageSize0} принимает в качестве параметра нулевое значение размера сообщения для передачи и приема. Вызывается функция инициализации модуля, проверяется, что она вернула \lstinline{GOOSE_INPUT_RET_CODE_INIT_ERROR} и система не сломалась.

Тест \lstinline{test_gooseInput_messageSize} принимает в качестве параметров корректные граничные значения размера сообщения для передачи и приема. Вызывается функция инициализации модуля, проверяется, что она вернула \lstinline{GOOSE_INPUT_RET_CODE_OK} и система работает корректно.

Тесты \lstinline{test_gooseInput_depth0} и \lstinline{test_gooseInput_depth} выполняются аналогично вышеописанным тестам с размером сообщения, только проверяется глубина очереди.

Есть группа тестов на приоритеты потоков, которых три в системе. Тесты основного потока \lstinline{test_prioProcTaskValid}, потока подписок \lstinline{test_prioSubscriptionTaskValid} и потока, ответственного за временные рамки работы системы, \lstinline{test_prioTimeProcessTaskValid} принимают граничные корректные значения приоритетов задач и ожидают, что вызываемая функция инициализации модуля вернет \lstinline{GOOSE_INPUT_RET_CODE_OK} и система работает корректно.

Тесты основного потока \lstinline{test_prioProcTaskInvalid}, потока подписок  \lstinline{test_prioSubscriptionTaskInvalid} и потока, ответственного за временные рамки работы системы, \lstinline{test_prioTimeProcessTaskInvalid} принимают некорректные значения приоритетов задач и ожидают, что вызываемая функция инициализации модуля вернет \lstinline{GOOSE_INPUT_RET_CODE_INIT_ERROR}.

\subsection{Тесты модуля \moduleProcessPackets}

При обработке пакета используются множество алгоритмов перед его непосредственным помещением в память. Как было указано ранее, пакеты могут содержать информацию с данными разных типов. Основной задачей этого раздела является тестирование всевозможных вариантов данных, их обработки и преобразования.

Все строки делятся на две группы: у которых длина больше 64 байт и у которых она меньше или равна этому значению. Внутри системы используются собственные типы данных, упрощающих работу модулей. На этапе тестирования достаточно знать, что длина сообщения соответствует одной из двух вышеописанных групп.

Тест  \lstinline{test_checkStringLess64Size} принимает пакеты со строковой информацией до 64 байт включительно, выполняет проверку преобразования данных, корректного разложения их память. Проверяется область памяти, куда должны были записаться новые данные. В случае успеха тест пройден корректно, в противном случае ожидается ошибка.

Тест  \lstinline{test_checkStringMore64Size} принимает пакеты со строковой информацией более 64 байт, выполняет проверку преобразования данных, корректного разложения их память. Проверяется область памяти, куда должны были записаться новые данные. В случае успеха тест пройден корректно, в противном случае ожидается ошибка.

Тесты для типов Integer, Boolean, Unsigned,
Dp, Quality и UtcTime выполняются аналогично вышеописанным тестам со строковыми типами данных. Также выполняется преобразования в типы системы и запись информации в память. Для текущих типов в модуле соответствуют конкретные аналоги, вариативность не поддерживается. При ошибке формата пришедших данных пакет отбрасывается. В случае успеха проверяется размер записанных в память данных, а также ее наполнение, которое должно соответствовать заявленному.

\subsection{Покрытие исходного кода}

Статистика помогает произвести оценку масштаба того или иного явления, а также разработать систему методов для анализа и изучения.

Тестовое покрытие -- это «плотность» покрытия тестами выполняемого программного кода или требований к нему. Чем больше тестовых вариантов проверено, тем выше процент покрытия. Выделяются следующие типы тестового покрытия:

\begin{itemize}
    \item покрытие требований;
    \item покрытие программного кода;
    \item проверочное покрытие на основе проанализированных данных потока управления.
\end{itemize}

Все типы тестового покрытия не только направлены на улучшение качества создания и проверки программного обеспечения, но и существенным образом повышают уровень удовлетворенности продуктом со стороны заказчика.

Для вычисления процента покрытия в данном проекте использовалась программа gcov, которая входит в состав набора инструментов GCC. Она сообщает, какой процент
строк исходного кода был проверен как для конкретной функции, так и для файла, целого модуля и даже системы, какие строки кода фактически выполнялись с контрольными примерами для достижения удовлетворительного охвата и ожидаемой работы, сколько времени длились вычисления для каждого фрагмента кода, что помогает в процессе оптимизации. Такого рода информация позволяет сделать выводы о необходимости более тщательного тестирования, если процент, возвращаемый программой, не соответствует
ожиданиям.

\begin{table}[ht]
    \caption{Статистика покрытия тестами системы}
    \label{table:testing:codeCoverageStats}
    \begin{tabular}{| >{\raggedright}m{0.6\textwidth}
                    | >{\centering\arraybackslash}m{0.347\textwidth}|}
        \hline
        \centering Название модуля & Покрытие, $\%$ \\

        \hline
        Модуль \moduleCfg & $ 91 $ \\

        \hline
        Модуль \moduleRecvPackets & $ 76 $ \\

        \hline
        Модуль \moduleProcessPackets & $ 84 $ \\

        \hline
    \end{tabular}
\end{table}

Высокие показатели в таблице~\ref{table:testing:codeCoverageStats} свидетельствуют об ответственном подходе к тестированию, грамотной постановке задачи для разработки системы. Однако результаты не равняются $ 100 \ \%$, что объясняется невозможностью или повышенной сложностью эмуляции работы функций-заглушек других модулей, невнимательностью тестировщиков, ограниченностью во времени работы, отказом заказчика от части тестов.
