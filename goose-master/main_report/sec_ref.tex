\referenceTitle

Дипломный проект предоставлен следующим образом. Электронные носители: 1
компакт-диск. Чертежный материал: 6 листов формата А1. Пояснительная записка:
\insertNumPagesText ,
\insertNumFiguresText , \insertNumTablesText ,
\insertNumBibElementsText ,
\insertNumAnnexesText .

Ключевые слова: GOOSE-пакет, \iecStd, \cid-файл, прием, обработка, реальное время, фильтрация, MAC-адрес.

Предметной областью данной разработки являются промышленные измерители
для релейной защиты и автоматики.
Объектом разработки является механизм получения и обработки данных протоколов стандарта \iecStd.

Целью данного дипломного проекта является разработать модуль приема и обработки GOOSE-пакетов как часть системы, которая способна работать как на операционной системе Linux, так и на встраиваемых решениях с использованием операционных систем реального времени.

Для разработки использовались язык программирования C,
система сборки CMake и пакетный менеджер Conan.
В качестве среды разработки
был взят редактор исходного кода Visual Studio Code.
Для тестирования применялись библиотеки Unity и CMock.

В результате был получен набор пакетов Conan с исходными кодами, которые
могут использоваться для обеспечения приема и обработки входящих
GOOSE-пакетов в соответствии со стандартом \iecStd.

Разработанная система проводит фильтрацию входных данных,
проверку пакета на целостность и корректность
структуры, извлекает полезную нагрузку
и сохраняет ее в виде, необходимом для дальнейшего
анализа и взаимодействия.
Система имеет возможность изменения конфигурации во время работы устройства
путем загрузки нового \cid-файла.

Разработка данного программного продукта является эффективной
и позволяет уменьшить затраты компании-заказчика.

Проект завершен и полностью готов к внедрению в коммерческие проекты.
Система имеет несколько направлений развития, которые могут быть реализованы
по требованию за приемлемые сроки.

\newpage
