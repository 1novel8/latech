% Contains config of surnames & other
% "~" - неразрывный пробел
% "\ " - Явно поставить пробел после команды
% Можно переиспользовать созданные выше переменные

% Даты, приказы и номера
\newcommand{\targetYear}{2023}
%% Приказ
\newcommand{\uniDecreeDate}{1 апреля \targetYear\ г.}
\newcommand{\uniDecreeNumber}{892-с}
\newcommand{\taskStartDate}{23 марта \targetYear\ г.}
\newcommand{\taskFinishDate}{1 июня \targetYear\ г.}
\newcommand{\diplomaVariant}{429}
\newcommand{\econCalcDate}{02.04.\targetYear\ г.}

% Люди
\newcommand{\studentShort}{Р.А.~Якунин}
\newcommand{\studentFullParental}{Якунин Роман Александрович}
\newcommand{\stdTestTutorShort}{Е.Е.~Клинцевич}
\newcommand{\headOfDepartmentShort}{Б.В.~Никульшин}
\newcommand{\practiceTutorShort}{А.В.~Бушкевич}
\newcommand{\practiceDepartmentTutorShort}{А.В.~Русакович}
\newcommand{\diplomaTutorShort}{\practiceDepartmentTutorShort}
\newcommand{\diplomaDepartmentTutorShort}{\practiceDepartmentTutorShort}
\newcommand{\diplomaEconomyTutorShort}{В.В.~Дершень}

% Названия
\newcommand{\taskNameFull}{Локальная компьютерная сеть}
\newcommand{\economicalPartName}{Технико-экономическое обоснование разработки модуля
    приема и обработки GOOSE-пакетов}
\newcommand{\rub}{р.}
\newcommand{\iec}{IEC}
\newcommand{\isoIec}{ISO/IEC}
\newcommand{\iso}{ISO}
\newcommand{\ieee}{IEEE}
\newcommand{\iecStd}{\iec\ 61850}
\newcommand{\libIec}{libiec61850}
\newcommand{\libXml}{libxml2}
% Также ссылаемся на нужную часть библиографии
\newcommand{\iecStdRef}[2]{\iecStd-{#1}-{#2}~\cite{IEC61850_#1_#2}}
\newcommand{\xml}{XML}
\newcommand{\cid}{ICD}
\newcommand{\xsd}{XSD}

% Схемы
\newcommand{\structScheme}{ГУИР.400201.\diplomaVariant \ С1}
\newcommand{\dataScheme}{ГУИР.400201.\diplomaVariant \ ПД.1}
\newcommand{\blockScheme}{ГУИР.400201.\diplomaVariant \ ПД.2}
\newcommand{\seqIcdScheme}{ГУИР.400201.\diplomaVariant \ РР.1}
\newcommand{\seqGooseScheme}{ГУИР.400201.\diplomaVariant \ РР.2}

% Модули структурного проектирования
\newcommand{\moduleCfg}{конфигурации}
\newcommand{\moduleXml}{обработки \xml}
\newcommand{\moduleSettingsApply}{применения настроек}
\newcommand{\moduleRecvPackets}{приема пакетов}
\newcommand{\moduleThreads}{распределения нагрузки}
\newcommand{\moduleProcessPackets}{обработки пакетов}
\newcommand{\moduleDataStoring}{хранения данных}
\newcommand{\moduleOsal}{обеспечения платформонезависимости}
\newcommand{\moduleLog}{протоколирования}

% Экономика
\FPeval{\configRoundSigns}{2}
\FPeval{\configPercentRoundSigns}{0}

% Остальное
\newcommand{\rubFormula}{\text{ \rub}}
