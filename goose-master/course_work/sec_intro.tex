\sectionCenteredToc{ВВЕДЕНИЕ}
\label{sec:intro}

В наше время каждое предприятие или компания имеет потребность в локальной компьютерной сети, что является неотъемлемым инструментом для обеспечения гармоничного взаимодействия всех сотрудников компании, обеспечения оперативного доступа к актуальной информации и эффективной совместной работы.

Целью этого курсового проекта является разработка локальной компьютерной сети для небольшой обувной компании. Ключевыми целями в ее разработке являются: реализация всех предопределенных заказчиком требований (бюджет, количество возможных подключений, предпочтения по безопасности и скорости), грамотная реализация архитектуры сети, стабильность и бесперебойность работы для комфортной работы каждого сотрудника.

Для реализации данного курсового проекта следует учесть следующие требования: архитектуру здания, а также плотность и ширину стен для расчета проходимости сигнала по всей площади помещения.
Стоит учесть количество стационарных пользователей, количество стационарных подключений, количество мобильных подключений, в том числе прочее оконечное оборудование.

Безопасность компьютерной локальной сети имеет первостепенное значение, поскольку она защищает конфиденциальные данные компании, предотвращает несанкционированный доступ и минимизирует риск утечек информации.
Важно обеспечить надежные меры защиты, чтобы сохранить репутацию, уверенность клиентов и предотвратить потенциальные проблемы.
Для проектирования данной компьютерной сети особое внимание следует уделить надежности хранения данных, а также стоит учесть требование заказчика в необходимости IPsec-VPN для удаленного подразделения.

При выборе сетевого оборудования будет использоваться производитель Cisco, который является одним из ведущих мировых производителей сетевого оборудования, обладая долгой историей и сильной репутацией.
Cisco предлагает широкий спектр продуктов и решений для сетевой инфраструктуры, что позволяет выбрать наилучшие варианты под текущие требования заказчика.

\begin{enumerate_num}
    \item Исследование и анализ протокола GOOSE.
    \item Проектирование модуля приема и обработки GOOSE-пакетов.
    \item Разработка блоков приема и обработки GOOSE-пакетов.
    \item Тестирование работоспособности модуля.
    \item Расчет экономических показателей дипломного проекта.
    \item Написание руководства пользователя.
\end{enumerate_num}
